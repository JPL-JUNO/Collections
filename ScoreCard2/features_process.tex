\chapter{特征处理}
\section{$\chi^2$分箱}
\subsection{$\chi^2$的计算}
\begin{equation}
    \chi^2=\sum_{i=i}^{2}\sum_{j=1}^{k}\frac{(A_{ij}-E_{ij})^2}{E_{ij}}
\end{equation}
这里 $k$ 是类的数量,$A_{ij}$ 是 $i$ 区间,$j$ 类中的数量,$R_i=\sum_{j=1}^{k}A_{ij}$是 $i$ 区间内的数量,$C_j=\sum_{i=1}^{2}A_{ij}$是 $j$ 类的数量,$N=\sum_{i=1}^{2}R_i$ 是所有样本的数量,$E_{ij}=R_i*C_j/N$ 是 $A_{ij}$ 期望频数。如果 $R_i$ 或 $C_j$ 为 0,则设置 $E_{ij}=0.1$。

\begin{table}
    \centering
    \caption{两个区间、两个类的$\chi^2$合并示意图}
    \begin{tabular}{rrrr}
        \hline
              & $y_0$      & $y_1$      & 合计          \\
        \hline
        $x_0$ & $A_{11}=a$ & $A_{12}=b$ & $R_1=a+b$   \\
        $x_1$ & $A_{21}=c$ & $A_{22}=d$ & $R_2=a+b$   \\
        合计    & $C_1=a+c$  & $C_2=b+d$  & $N=a+b+c+d$ \\
        \hline
    \end{tabular}
\end{table}
\begin{equation}
    \begin{aligned}
        E_{ij} & =
        \begin{bmatrix}
            \frac{(a+b)(a+c)}{a+b+c+d} & \frac{(a+b)(b+d)}{a+b+c+d} \\
            \frac{(c+d)(a+c)}{a+b+c+d} & \frac{(c+d)(b+d)}{a+b+c+d} \\
        \end{bmatrix}=\begin{bmatrix}
                          \frac{(a+b)(a+c)}{N} & \frac{(a+b)(b+d)}{N} \\
                          \frac{(c+d)(a+c)}{N} & \frac{(c+d)(b+d)}{N} \\
                      \end{bmatrix} \\
               & =\frac{1}{N}
        \begin{bmatrix}
            a+b \\
            c+d \\
        \end{bmatrix}
        \begin{bmatrix}
            a+c & b+d \\
        \end{bmatrix}=\frac{1}{N}\bm{R}\bm{J}
    \end{aligned}
\end{equation}
这样代码中可以直接使用广播机制来计算$E_{ij}$